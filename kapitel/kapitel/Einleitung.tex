\newpage
\section{Einleitung}
\subsection{Ein Medienecho}
https://www.computerwoche.de/a/was-gegen-cloud-native-spricht,3613754
https://www.datacenter-insider.de/podcastmitsebastiankistervomkubernetes-kompetenz-zentrumbeiaudi-a-6276da9a37b65eed7af33b32199ecdbd/
https://www.it-business.de/cloud-native-ist-im-channel-angekommen-a-ea90aa27d988c6c0c5815aaacc56d360/
https://www.security-insider.de/cloud-native-security-vier-herausforderungen-und-drei-tipps-a-51f23e225b38678ae8ec8b38e10941ed/

Cloud Computing, 
\subsection{Problemstellung und Zielsetzung der Thesis}
\subsubsection{problem}
Die rasante Entwicklung von Cloud-Technologien und die wachsende Nachfrage nach agilen und skalierbaren Anwendungen haben zu einer Revolution in der Art und Weise geführt, wie Unternehmen ihre Software entwickeln, bereitstellen und betreiben. Die traditionelle monolithische Architektur wird zunehmend von einer auf Microservices basierenden Architektur abgelöst, die es ermöglicht, Anwendungen in kleinere, eigenständige Komponenten zu zerlegen. Diese Komponenten, auch als containerisierte Microservices bezeichnet, können unabhängig voneinander entwickelt, bereitgestellt und skaliert werden. Die Cloud bietet eine ideale Umgebung für den Einsatz solcher Microservices, da sie elastische Ressourcen, automatische Skalierung und flexible Bereitstellungsmöglichkeiten bietet.

Der Einsatz von containerisierten Microservices in der Cloud bringt jedoch auch neue Herausforderungen mit sich. Eine entscheidende Frage ist die Auswahl geeigneter Technologien und Tools, um die Microservices effizient zu verwalten und zu orchestrieren. Container-Orchestrierungssysteme wie Kubernetes haben sich als Standard etabliert, bieten jedoch eine Vielzahl von Möglichkeiten und Konfigurationen, die sorgfältig abgewogen werden müssen. Die Gewährleistung von Skalierbarkeit ist ein weiterer wichtiger Aspekt. Eine erfolgreiche Cloud-Native-Plattform muss in der Lage sein, die Anwendungen dynamisch zu skalieren, um eine hohe Auslastung zu bewältigen und gleichzeitig Ressourcenverschwendung zu vermeiden.

Darüber hinaus dürfen die Sicherheitsaspekte nicht vernachlässigt werden. Eine Cloud-Native-Plattform muss Mechanismen zur Isolation und Absicherung der einzelnen Microservices bereitstellen, um die Vertraulichkeit, Integrität und Verfügbarkeit der Daten zu gewährleisten. Ebenso spielt die effiziente Ressourcennutzung eine zentrale Rolle, da eine effektive Verwaltung und Auslastung der Ressourcen in der Cloud entscheidend für die Wirtschaftlichkeit und Leistungsfähigkeit der Plattform ist.

Vor diesem Hintergrund wird deutlich, dass der architektonische Entwurf und die Modellierung einer geeigneten Cloud-Native-Plattform von entscheidender Bedeutung sind. Eine fundierte und gut durchdachte Architektur legt den Grundstein für den erfolgreichen Einsatz von containerisierten Microservices und Anwendungen in der Cloud. Durch die Schaffung einer skalierbaren, sicheren und effizienten Plattform können Unternehmen die Vorteile der Cloud voll ausschöpfen und ihre Anwendungen mit Agilität und Skalierbarkeit entwickeln und betreiben.
\\
\subsubsection{ziel}
Das Ziel dieser Arbeit ist es, einen umfassenden architektonischen Entwurf und eine beispielhafte Modellierung einer Cloud-Native-Plattform für den erfolgreichen Einsatz von containerisierten Microservices und Anwendungen zu entwickeln. Dabei sollen nicht nur die Anforderungen an Skalierbarkeit und effiziente Ressourcennutzung berücksichtigt werden, sondern auch weitere wichtige Aspekte, die für den reibungslosen Betrieb der Plattform von Bedeutung sind.

Ein zentraler Fokus liegt auf der Auswahl geeigneter Technologien, die eine nahtlose Integration und Interaktion der Microservices ermöglichen. Hierbei werden verschiedene Aspekte berücksichtigt, wie die Containerisierungstechnologie (z. B. Docker), die Orchestrierung (z. B. Kubernetes), das Service-Discovery-Management und die Konfigurationsverwaltung. Die Auswahl der richtigen Technologien spielt eine entscheidende Rolle bei der Gewährleistung von Skalierbarkeit, Flexibilität und Wartbarkeit der Plattform.

Des Weiteren wird die Integration von Monitoring-, Logging- und Tracing-Funktionalitäten in den architektonischen Entwurf und die Modellierung einbezogen. Dies ist von großer Bedeutung, um eine umfassende Überwachung und Analyse der Microservices und deren Kommunikation zu ermöglichen. Durch die Implementierung dieser Funktionen können Performance-Probleme, Engpässe und Fehler frühzeitig erkannt und behoben werden.

Darüber hinaus wird in der Arbeit auch der Sicherheitsaspekt berücksichtigt. Die Architektur und Modellierung der Cloud-Native-Plattform müssen Mechanismen zur Sicherung der Datenintegrität, zum Schutz vor unbefugtem Zugriff und zur Abwehr potenzieller Bedrohungen umfassen. Hierbei werden verschiedene Sicherheitsmaßnahmen wie Authentifizierung, Autorisierung und Verschlüsselung in den Entwurf integriert.

Durch den erarbeiteten architektonischen Entwurf und die Modellierung einer Cloud-Native-Plattform wird eine solide Grundlage geschaffen, auf der Unternehmen aufbauen können, um die Vorteile von containerisierten Microservices und Anwendungen in der Cloud optimal zu nutzen und ihre IT-Infrastruktur effektiv zu modernisieren. Die entwickelte Plattform ermöglicht eine effiziente Ressourcennutzung, Skalierbarkeit und Flexibilität und unterstützt Unternehmen dabei, ihre Anwendungen agil zu entwickeln, zu betreiben und auf zukünftige Anforderungen anzupassen.
\subsection{Abgrenzung}
Die Festlegung der Abgrenzung der vorliegenden Thesis ist von entscheidender Bedeutung, um den Umfang der Arbeit klar zu definieren und den Fokus gezielt auf die Themenspezifischen Aspekte des architektonischen Entwurfs und der Modellierung einer Cloud-Native-Plattform für containerisierte Microservices und Anwendungen zu lenken. Durch die klare Definition des Umfangs werden die Grenzen der Untersuchung im weitreichenden Cloud-Native Themenkomplex abgesteckt und die Aufmerksamkeit auf die relevanten Themenbereiche gelenkt, die im Kontext dieser Arbeit von Interesse sind.
\begin{itemize}
	\item  \textbf{Technologische Ausrichtung:} Die Hausarbeit konzentriert sich auf Cloud-Native-Technologien und -Ansätze, insbesondere auf die Verwendung von Containern und Microservices. Andere Ansätze, wie beispielsweise virtuelle Maschinen oder herkömmliche monolithische Architekturen, werden nicht im Detail behandelt.
	\item \textbf{Plattformfokus:} Die Arbeit legt den Schwerpunkt auf die Entwicklung einer Cloud-Native-Plattform, die den spezifischen Anforderungen von containerisierten Microservices und Anwendungen gerecht wird. Dabei werden Aspekte wie Architekturdesign, Auswahl geeigneter Technologien und Integration von Monitoring-, Logging- und Tracing-Funktionalitäten berücksichtigt. Die Implementierung und konkrete Umsetzung der Plattform in einer bestimmten Cloud-Umgebung oder mit spezifischen Tools wird jedoch nicht im Detail behandelt.
	\item \textbf{Anwendungsbereich:} Die Hausarbeit fokussiert sich auf den generellen architektonischen Entwurf und die Modellierung einer Cloud-Native-Plattform für containerisierte Microservices und Anwendungen. Es werden keine spezifischen Anwendungsdomänen oder Industrien betrachtet. Die vorgeschlagene Architektur und Modellierung sollten jedoch auf verschiedene Anwendungsfälle und Branchen anwendbar sein.
	\item \textbf{Zeitliche Betrachtung:} Die Hausarbeit bezieht sich auf den aktuellen Stand der Technologie und Best Practices zum Zeitpunkt der Erstellung der Arbeit. Zukünftige Entwicklungen oder Trends im Bereich der Cloud-Native-Architektur und Containerisierung können nicht berücksichtigt werden.
\end{itemize}

\subsection{Methodik und Vorgehensweise}