\newpage
\section{Einleitung}\label{lab:einleitung}
Die fortschreitende Digitalisierung und die rasante Entwicklung von Technologien haben die Art und Weise, wie Softwareanwendungen entwickelt und bereitgestellt werden, grundlegend verändert. In der heutigen Arbeitswelt ist Agilität, Skalierbarkeit und Effizienz entscheidend für den Erfolg von Unternehmen. Aus diesem Grund haben sich neben zeitgemäßen Softwarearchitekturen und Technologien auch innovative Arbeitsmethoden und Vorgehensmodelle als grundlegende Elemente zeitgemäßer Softwareentwicklung und Bereitstellung etabliert.

Die Weltwirtschaft steht vor einer Ära, in der die nahtlose Bereitstellung von Diensten und Anwendungen unabhängig von räumlichen Einschränkungen von höchster Priorität ist. Die jüngsten Ereignisse, wie die globale COVID-19-Pandemie, haben die Notwendigkeit unterstrichen, dass Unternehmen agil und flexibel auf sich ändernde Marktbedingungen reagieren können. Laut einem Bericht von McKinsey hat die Pandemie die digitale Transformation beschleunigt und die Nachfrage nach Cloud-basierten Lösungen verstärkt, um remote zu arbeiten, Geschäftsprozesse umzugestalten und Kunden digital zu erreichen \footcite[Vgl.][o.S.]{McKinsey2020}.



In dieser dynamischen Umgebung ist es von entscheidender Bedeutung, die Methoden und Ansätze zu verstehen, die bei der Gestaltung und Modellierung von Cloud-Native Plattformen für containerisierte Anwendungen angewendet werden können. Die vorliegende Thesis widmet sich genau diesem Thema und beabsichtigt, einen tiefgreifenden Einblick in den architektonischen Entwurf und die Modellierung solcher Plattformen zu bieten. Indem sie aktuelle Nachrichten, Trends und bewährte Praktiken berücksichtigt, strebt die Thesis danach, einen Beitrag zur Weiterentwicklung dieser bedeutenden Technologie zu leisten und Unternehmen bei der Realisierung ihrer digitalen Visionen zu unterstützen.
%

% https://books.google.de/books?hl=de&lr=lang_de&id=M8BSEAAAQBAJ&oi=fnd&pg=PR5&dq=cloud+native+vor+und+nachteile&ots=PMqYIvEfhg&sig=ZQG0HF_mhcHQeJHsJsinoYrmyzw&redir_esc=y#v=onepage&q=cloud%20native%20vor%20und%20nachteile&f=false

% Market Shares der CLoud Anbieter
%https://de.statista.com/statistik/daten/studie/150979/umfrage/marktanteile-der-fuehrenden-unternehmen-im-bereich-cloud-computing/


\subsection{Ein Medienecho}\label{lab:medienecho}
Der weitreichende und stetig wachsende Themenbereich der Cloud-Technologien hat über die letzten Jahre hinweg eine kontinuierliche Präsenz in einer Vielzahl von Medienkanälen aufrechterhalten, darunter Fachzeitschriften, Online-Blogs Podcasts sowie verschiedenste andere Formen von Medienplattformen und -Formaten. So behandelt der Artikel von David Linthicum eine skeptische Sicht auf eine von Gartner veröffentlichte Studie, in welcher prognostiziert wird dass bis 2025 mehr als 95\% von Anwendungs-Workloads in einer Cloud Nativen Plattform laufen werden. Grundlegend möchte der Autor hervorheben, dass Unternehmen bei neuen Technologien neben Chancen auch die Risiken berücksichtigt sollen\footcite[Vgl.][o.S.]{compWoche}. So werden im Artikel drei Aspekte betont:
\begin{itemize}
	\item Vendor Lock-in: Anwendungen die gezielt für eine bestimmt Cloud Plattformen entwickelt wurden, lassen sich schwieriger auf andere Plattformen übertragen. Die eingeschränkte Portabilität steht somit zum Teil im Widerspruch dessen was Cloud-Native Anwendungen definiert.
	\item Skill Gap: Unternehmen ohne Erfahrung stehen vor Herausforderungen, die zusätzliche Schulungen oder Ressourcen erfordern, was zu schlecht konzipierten oder übermäßig komplexen Anwendungen führen kann. Dies wiederum könnte die Effizienz beeinträchtigen und möglicherweise die gesamte Umsetzung gefährden.
	\item unkontrollierter Kostenanstieg: Die nutzungsabhängige Preisgestaltung kann zu unvorhergesehenen Mehrkosten führen, wenn Anwendungen plötzlich stark frequentiert werden. 
\end{itemize}

Pokemon Go, ein Spiel welches 2016 für Android und iOS Gräte erschien also für Smartphones und Tablets, setzt ebenfalls auf die Cloud Technologie von Google. Im News Blog von Google wird dargelegt wie es den Entwicklern möglich war mithilfe der bereitgestellten Cloud Technologie Live-Events im Spiel mit einem Transaktionsvolumen von 400.000 bist fast zu einer Millionen Transaktionen pro Sekunde umzugehen. Weiterhin wird ausgeführt, dass im Backend der Infrastruktur  Services flexibel, nach bedarf skalieren. Täglich werden 5-10 Terabyte an Daten im Rahmen Datenanalysen verarbeitet. Außerdem wird hervorgehoben, dass die Stabilität und Gesundheit durch umfassendes Logging, Monitoring und umfangreiche Dashboards sichergestellt wird \footcite[Vgl.][o.S.]{pokemonGo}.
\subsection{Problemstellung und Zielsetzung der Thesis}
Die rasante Entwicklung von Cloud-Technologien und die wachsende Nachfrage nach agilen und skalierbaren Anwendungen haben zu einer Revolution in der Art und Weise geführt, wie Unternehmen ihre Software entwickeln, bereitstellen und betreiben. Die traditionelle monolithische Anwendungsarchitektur wird zunehmend von einer auf Microservices basierenden Architektur abgelöst, die es ermöglicht, Anwendungen in kleinere, eigenständige Komponenten zu zerlegen. Diese als containerisierte Microservices\footcite[Vgl.][S.4]{wolff2018microservices} bezeichneten Software Komponenten, können unabhängig voneinander entwickelt, bereitgestellt und skaliert werden. Cloudlösungen werben damit, eine ideale Umgebung für den Einsatz solcher Microservices zu bieten, da sie unter anderem Ressourcen elastische bereitstellen und automatische skalieren können\footcite[Vgl.][S.8--19]{henneberger2016cloud}.

Der Einsatz von containerisierten Microservices in der Cloud bringt jedoch auch neue Herausforderungen mit sich. Wie das Wort Microservice bereits indiziert, liegt die Vermutung nahe, dass in einer Anwendungslandschaft zahlreiche weitere Microservices existieren, die als Konglomerat einen ganzheitlichen Service bereitstellen. Für den Betrieb im Produktiven Umfeld bedeutet dementsprechend, dass der Monitoring aufwand für Microservices steigt. Eine entscheidende Frage ist die Auswahl geeigneter Technologien und Tools, um die Microservices effizient zu verwalten und orchestrieren\footcite[Vgl.][S.12-16]{wolff2018microservices}.
Container-Orchestrierungssysteme wie Kubernetes haben sich als Standard etabliert, bieten zudem eine Vielzahl von Möglichkeiten und Konfigurationen, die sorgfältig abgewogen werden müssen. Die Gewährleistung von Skalierbarkeit ist ein weiterer essenzieller Aspekt im Cloud Kontext.  Um hohe Workloads effizient zu bewältigen und Ressourcenverschwendung zu vermeiden, muss eine Cloud-Native-Plattform in der Lage sein, Anwendungen dynamisch zu skalieren. Eine effektive Verwaltung und Auslastung der Ressourcen in der Cloud sind entscheidend für die Wirtschaftlichkeit und Leistungsfähigkeit der Plattform.

%Darüber hinaus sind die Sicherheitsaspekte nicht zu vernachlässigen. Eine Cloud-Native-Plattform muss Mechanismen zur Isolation und Absicherung der einzelnen Microservices bereitstellen, um die Vertraulichkeit, Integrität und Verfügbarkeit der Daten zu gewährleisten.

Vor diesem Hintergrund wird deutlich, dass der architektonische Entwurf und die Modellierung einer geeigneten Cloud-Native-Plattform von entscheidender Bedeutung sind. Eine fundierte und gut durchdachte Architektur legt den Grundstein für den erfolgreichen Einsatz von containerisierten Microservices und Anwendungen in der Cloud. Durch die Schaffung einer skalierbaren, sicheren und effizienten Plattform können Unternehmen die Vorteile der Cloud voll ausschöpfen und ihre Anwendungen mit Automation, Agilität und Skalierbarkeit entwickeln und betreiben.

Die Thesis stellt sich die Forschungsfrage: "Wie kann ein architektonischer Entwurf für eine Cloud-Native-Plattform aussehen um den neuen Herausforderungen in einer Nativen Cloud Umgebung zu begegnen?\" zu begegnen eine Modellierung einer Cloud-Native-Plattform realisieren, die containerisierte Microservices und Anwendungen optimal unterstützt und dabei Skalierbarkeit, effiziente Ressourcennutzung und Application Observability in einem ausgewogenen Verhältnis berücksichtigt, um einen reibungslosen Betrieb und hohe Leistung zu gewährleisten?

Das Ziel dieser Thesis ist es, einen architektonischen Entwurf und eine daraus folgende beispielhafte Modellierung einer Cloud-Native-Plattform für den Einsatz von containerisierten Microservices und Anwendungen zu entwickeln. Dabei sollen nicht nur die Anforderungen an Skalierbarkeit und effiziente Ressourcennutzung berücksichtigt werden, sondern auch weitere Aspekte im Sinne der Application Observability, die für den reibungslosen Betrieb der Plattform und den darauf ansässigen Anwendungen von Bedeutung sind.

Ein zentraler Fokus liegt auf der Auswahl geeigneter Technologien, die eine nahtlose Integration und Interaktion der Microservices ermöglichen. Hierbei werden verschiedene Aspekte berücksichtigt, wie die Containerisierungstechnologie (z. B. Docker), die Orchestrierung (z. B. Kubernetes), das Service-Discovery-Management und die Konfigurationsverwaltung. Die Auswahl der richtigen Technologien spielt eine entscheidende Rolle bei der Gewährleistung von Skalierbarkeit, Flexibilität und Wartbarkeit der Plattform.

Des Weiteren wird die Integration von Monitoring-, Logging- und Tracing-Funktionalitäten in den architektonischen Entwurf und die Modellierung einbezogen. Dies ist von großer Bedeutung, um eine umfassende Überwachung und Analyse der Microservices und deren Kommunikation zu ermöglichen. Durch die Implementierung dieser Funktionen können Performance-Probleme, Engpässe und Fehler frühzeitig erkannt und behoben werden.

Darüber hinaus wird in der Arbeit auch der Sicherheitsaspekt berücksichtigt. Die Architektur und Modellierung der Cloud-Native-Plattform müssen Mechanismen zur Sicherung der Datenintegrität, zum Schutz vor unbefugtem Zugriff und zur Abwehr potenzieller Bedrohungen umfassen. Hierbei werden verschiedene Sicherheitsmaßnahmen wie Authentifizierung, Autorisierung und Verschlüsselung in den Entwurf integriert.

Durch den erarbeiteten architektonischen Entwurf und die Modellierung einer Cloud-Native-Plattform wird eine solide Grundlage geschaffen, auf der Unternehmen aufbauen können, um die Vorteile von containerisierten Microservices und Anwendungen in der Cloud optimal zu nutzen und ihre IT-Infrastruktur effektiv zu modernisieren. Die entwickelte Plattform ermöglicht eine effiziente Ressourcennutzung, Skalierbarkeit und Flexibilität und unterstützt Unternehmen dabei, ihre Anwendungen agil zu entwickeln, zu betreiben und auf zukünftige Anforderungen anzupassen.

[Zusammenfassung Problemstellung, Forschungsfrage, Zielsetzung der Thesis]

\subsection{Abgrenzung}
Die Festlegung der Abgrenzung der vorliegenden Thesis ist von entscheidender Bedeutung, um den Umfang der Arbeit klar zu definieren und den Fokus gezielt auf die Themenspezifischen Aspekte des architektonischen Entwurfs und der Modellierung einer Cloud-Native-Plattform für containerisierte Microservices und Anwendungen zu lenken. Durch die klare Definition des Umfangs werden die Grenzen der Untersuchung im weitreichenden Cloud-Native Themenkomplex abgesteckt und die Aufmerksamkeit auf die relevanten Themenbereiche gelenkt, die im Kontext dieser Arbeit von Interesse sind.
\begin{itemize}
	\item  \textbf{Technologische Ausrichtung:} Die Hausarbeit konzentriert sich auf Cloud-Native-Technologien und -Ansätze, insbesondere auf die Verwendung von Containern und Microservices. Andere Ansätze, wie beispielsweise virtuelle Maschinen oder herkömmliche monolithische Architekturen, werden nicht im Detail behandelt.
	\item \textbf{Plattformfokus:} Die Arbeit legt den Schwerpunkt auf die Entwicklung einer Cloud-Native-Plattform, die den spezifischen Anforderungen von containerisierten Microservices und Anwendungen gerecht wird. Dabei werden Aspekte wie Architekturdesign, Auswahl geeigneter Technologien und Integration von Monitoring-, Logging- und Tracing-Funktionalitäten berücksichtigt. Die Implementierung und konkrete Umsetzung der Plattform in einer bestimmten Cloud-Umgebung oder mit spezifischen Tools wird jedoch nicht im Detail behandelt.
	\item \textbf{Anwendungsbereich:} Die Hausarbeit fokussiert sich auf den generellen architektonischen Entwurf und die Modellierung einer Cloud-Native-Plattform für containerisierte Microservices und Anwendungen. Es werden keine spezifischen Anwendungsdomänen oder Industrien betrachtet. Die vorgeschlagene Architektur und Modellierung sollten jedoch auf verschiedene Anwendungsfälle und Branchen anwendbar sein.
	\item \textbf{Zeitliche Betrachtung:} Die Hausarbeit bezieht sich auf den aktuellen Stand der Technologie und Best Practices zum Zeitpunkt der Erstellung der Arbeit. Zukünftige Entwicklungen oder Trends im Bereich der Cloud-Native-Architektur und Containerisierung können nicht berücksichtigt werden.
\end{itemize}

[exkludiert sind Managed Services wie (AWS, GKE, etc.) Zusammenfassung der Abgrenzung]

\subsection{Methodik und Vorgehensweise}
Literaturrecherche, Modellentwicklung auf Basis der Literatur, Anforderungsanalyse, Prototyp Entwicklung in Form einer praxis Implementierung