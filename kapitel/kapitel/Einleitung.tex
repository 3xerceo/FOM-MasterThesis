\newpage
\section{Einleitung}\label{lab:einleitung}
Die fortschreitende Digitalisierung und die rasante Entwicklung von Technologien haben die Art und Weise, wie Softwareanwendungen entwickelt und bereitgestellt werden, grundlegend verändert. In der heutigen Arbeitswelt ist Agilität, Skalierbarkeit und Effizienz entscheidend für den Erfolg von Unternehmen. Aus diesem Grund haben sich neben zeitgemäßen Softwarearchitekturen und Technologien auch innovative Arbeitsmethoden und Vorgehensmodelle als grundlegende Elemente zeitgemäßer Softwareentwicklung und Bereitstellung etabliert. Die Weltwirtschaft steht vor einer Ära, in der die nahtlose Bereitstellung von Diensten und Anwendungen unabhängig von räumlichen Einschränkungen von höchster Priorität ist. Die jüngsten Ereignisse, wie die globale COVID-19-Pandemie, haben die Notwendigkeit unterstrichen, dass Unternehmen agil und flexibel auf sich ändernde Marktbedingungen reagieren können. Laut einem Bericht von McKinsey hat die Pandemie die digitale Transformation beschleunigt und die Nachfrage nach Cloud-basierten Lösungen verstärkt, um aus dem HomeOffice heraus zu arbeiten, Geschäftsprozesse umzugestalten und Kunden digital zu erreichen \footcite[Vgl.][o.S.]{McKinsey2020}.

In einer dynamischen Cloud Umgebung ist es von signifikanter Relevanz, die Methoden und Ansätze zu verstehen, die bei der Gestaltung und Modellierung von \cn Plattformen für containerisierte Anwendungen angewendet werden können. Die vorliegende Thesis widmet sich genau diesem Thema und beabsichtigt, einen tiefgreifenden Einblick in den architektonischen Entwurf und die Modellierung solcher Plattformen zu bieten. Indem sie aktuelle Nachrichten, Trends und bewährte Praktiken berücksichtigt, strebt die Thesis danach, einen Beitrag zur Weiterentwicklung dieser marktaktuellen Technologie zu leisten und Leser bei der Realisierung ihrer digitalen Visionen zu unterstützen.


\subsection{Ein Medienecho}\label{lab:medienecho}
Der weitreichende und stetig wachsende Themenbereich der Cloud-Technologien hat über die letzten Jahre hinweg eine kontinuierliche Präsenz in einer Vielzahl von Medienkanälen aufrechterhalten, darunter Fachzeitschriften, Online-Blogs, Podcasts sowie verschiedenste andere Formen von Medienplattformen und -Formaten. So behandelt der Artikel von David Linthicum eine skeptische Sicht auf eine von Gartner veröffentlichte Studie, in welcher prognostiziert wird, dass bis 2025 mehr als 95\% von Anwendungs-Workloads in einer \cn Plattform laufen werden. Grundlegend möchte der Autor hervorheben, dass Unternehmen bei neuen Technologien neben Chancen auch die Risiken berücksichtigt sollen\footcite[Vgl.][o.S.]{compWoche}. So werden im Artikel drei Aspekte betont:
\begin{itemize}
	\item Vendor Lock-in: Anwendungen die gezielt für eine bestimmt Cloud Plattformen entwickelt wurden, lassen sich schwieriger auf andere Plattformen übertragen. Die eingeschränkte Portabilität steht somit zum Teil im Widerspruch dessen was \cn Anwendungen definiert.
	\item Skill Gap: Unternehmen ohne Erfahrung stehen vor Herausforderungen, die zusätzliche Schulungen oder Ressourcen erfordern, was zu schlecht konzipierten oder übermäßig komplexen Anwendungen führen kann. Dies wiederum könnte die Effizienz beeinträchtigen und möglicherweise die gesamte Umsetzung gefährden.
	\item Unkontrollierter Kostenanstieg: Die nutzungsabhängige Preisgestaltung kann zu unvorhergesehenen Mehrkosten führen, wenn Anwendungen plötzlich stark frequentiert werden. 
\end{itemize}

Pokemon Go, ein Spiel welches 2016 für Android und iOS Gräte wie Smartphones und Tablets erschien, setzt ebenfalls auf die Cloud Technologie von Google. Im News Blog von Google wird dargelegt wie es den Entwicklern möglich war mithilfe der bereitgestellten Cloud Technologie Live-Events im Spiel mit einem Transaktionsvolumen von 400.000 bis fast zu einer Millionen Transaktionen pro Sekunde umzugehen. Weiterhin wird ausgeführt, dass im Backend der Infrastruktur  Services flexibel, nach Bedarf skalieren. Täglich werden 5-10 Terabyte an Daten im Rahmen Datenanalysen verarbeitet. Außerdem wird hervorgehoben, dass die Stabilität und Gesundheit durch umfassendes Logging, Monitoring und umfangreiche Dashboards sichergestellt wird \footcite[Vgl.][o.S.]{pokemonGo}.

\subsection{Problemstellung und Zielsetzung der Thesis}\label{lab:Problem}
Die traditionelle monolithische Anwendungsarchitektur wird zunehmend von einer auf Microservices\footcite[Vgl.][S.4]{wolff2018microservices} basierenden Architektur abgelöst, die es ermöglicht, Anwendungen in kleinere, eigenständige Komponenten zu zerlegen. Diese als containerisierte Microservices bezeichneten Software Komponenten, können unabhängig voneinander entwickelt, bereitgestellt und skaliert werden. Cloudlösungen werben damit, eine ideale Umgebung für den Einsatz solcher Microservices zu bieten, da sie unter anderem Ressourcen elastische bereitstellen und automatische skalieren können\footcite[Vgl.][S.8-19]{henneberger2016cloud}.

Der Einsatz von containerisierten Microservices in der Cloud bringt jedoch auch neue Herausforderungen mit sich. Wie das Wort Microservice bereits indiziert, liegt die Vermutung nahe, dass in einer Anwendungslandschaft zahlreiche weitere Microservices existieren, die als Konglomerat einen ganzheitlichen Service bereitstellen. Für den Betrieb im Produktiven Umfeld bedeutet dementsprechend, dass der Monitoring aufwand für Microservices steigt. Eine entscheidende Frage ist die Auswahl geeigneter Technologien und Tools, um die Microservices effizient zu verwalten und orchestrieren\footcite[Vgl.][S.12-16]{wolff2018microservices}.
Container-Orchestrierungssysteme wie Kubernetes haben sich als Industriestandard etabliert und bieten zudem eine Vielzahl an Konfigurationsmöglichkeiten. Die Gewährleistung von Skalierbarkeit ist ein weiterer essenzieller Aspekt im Cloud Kontext. Um dynamische Workloads effizient zu bewältigen und Ressourcenverschwendung zu vermeiden, muss eine \cn Plattform in der Lage sein, Anwendungen flexibel zu skalieren. Eine effektive Verwaltung und Auslastung der Ressourcen in der Cloud sind entscheidend für die Wirtschaftlichkeit und Leistungsfähigkeit der Plattform\footcite[Vgl.][o.S.]{k8s2022}.

Die Thesis beschäftigt sich mit der Forschungsfrage: \glqq Wie kann ein architektonischer Entwurf für eine \cn Plattform aussehen um den neuen Herausforderungen in einer \cn Umgebung zu begegnen\grqq? Hauptaugenmerk liegt im Vergleich zu klassischen Architekturen nicht auf der Hardware Infrastruktur und der Ressourcen Ausstattung wie \ac{CPU}, \ac{RAM} und Speicherplatz, sondern Aspekte im Sinne der Application Observability, die für den reibungslosen Betrieb der Plattform und den darauf ansässigen Anwendungen von Belang sind. Weiterhin betrachtet die Thesis eine eigenständig administrierte, on-Premise betriebene Plattform und keinen Dienstleistungsbezug von einem Service Provider in Form einer \ac{SaaS} oder \ac{PaaS} Lösung.

Das vorrangige Ziel dieser Forschungsarbeit besteht darin, eine eingehende Untersuchung des Themenfelds Application Observability im Kontext von \cn Anwendungen durchzuführen, wobei besonderes Augenmerk auf die Identifizierung und Analyse relevanter Aspekte gelegt wird, die für den effizienten Betrieb von Anwendungen relevant sind. Darüber hinaus beabsichtigt diese Arbeit, die gewonnenen Erkenntnisse und Einsichten in ein umfassendes Architekturmodell zu überführen, das als Leitfaden und Rahmenwerk für die Implementierung und das Management von Application Observability in Cloud Native Umgebungen dienen soll.

\subsection{Abgrenzung}\label{lab:Abgrenzung}
Die Herausforderungen im Bereich der Application Observability von \cn Anwendungen sind vielfältig. Angesichts dieser Komplexität ist es unabdingbar, eine klare Abgrenzung und Fokussierung für die vorliegende Arbeit festzulegen. Die Themen, die im Rahmen dieser Thesis im Kontext der Application Observability untersucht werden sind\footcite[Vgl.][S.5]{Pourmajidi2023}:
\begin{itemize}
	\item Monitoring: Die Erfassung von Metriken von Hardware nahen Ressourcen wie CPU und RAM. Aber auch anwendungsspezifische Metriken wie den Status von Diensten und Echtzeitinformationen der Anwendung. Die Thesis wird sich darauf konzentrieren, wie Monitoring-Lösungen in Cloud-Native Umgebungen implementiert werden können, um einen umfassenden Einblick in die Gesundheit der Anwendungen zu gewährleisten.
	\item Tracing: Tracing von Anfragen oder Transaktionen durch verschiedene Microservices in \cn Anwendungen wird betrieben um Engpässe und Latenzprobleme zu identifizieren. Diese Arbeit wird die Konzepte und Tools zur Tracing-Implementierung in Cloud-Native Umgebungen beleuchten und zeigen, wie sie dazu beitragen, die Leistung und Zuverlässigkeit der Anwendungen zu verbessern.
	\item Logging: Das Sammeln und Analysieren von Logs ist ein wesentlicher Bestandteil der Application Observability. Logs liefern Einblicke in Fehler, Ausnahmen und Aktivitäten in Anwendungen. Die Thesis wird sich mit bewährten Praktiken für das Log-Management in Cloud-Native Anwendungen befassen, einschließlich der Strukturierung von Logdaten und deren Integration in zentrale Log-Aggregationsplattformen.
	\item Telemetriedaten: Telemetriedaten umfassen eine Vielzahl von Informationen, die zur Verbesserung der Anwendungsleistung beitragen können. Dies schließt zum Beispiel Netzwerkkommunikation oder, Datenbankzugriffe ein. Die Arbeit wird die Erfassung von Telemetriedaten in Form der Netzwerkkommunikation in Cloud-Native Umgebungen behandeln und visualisieren.
\end{itemize}
Des weiteren werden Dienstleistungen großer Cloud Plattformanbieter nicht betrachtet.

\subsection{Methodik und Vorgehensweise}\label{lab:methodik}

Die vorliegende Arbeit basiert auf einer umfassenden Recherche und Erfassung bereits vorhandener wissenschaftlicher Literatur, um einen grundlegenden theoretischen Rahmen zu schaffen. Um die Forschungsfrage zu beantworten werden gewonnene Erkenntnisse aus der Literatur dafür verwendet ein Modell abzuleiten, welches den Herausforderungen im \cn Umfeld unter Hilfenahme der Aspekte aus Application Observability begegnet.

%Literaturrecherche, Modellentwicklung auf Basis der Literatur, Anforderungsanalyse, Prototyp Entwicklung in Form einer praxis Implementierung

Die Struktur dieser Arbeit gliedert sich in einen theoretischen und einen praktischen Teil. Kapitel \ref{lab:einleitung} dient der thematischen Einführung. Kapitel \ref{lab:softwarearchitektur} bietet einen theoretischen Überblick über die relevanten Themen Softwarearchitektur. Hier werden die Begriffe definiert und eingegrenzt. Speziell betrachtet werden im Kapitel der Softwarearchitektur Modelle wie 3-Tier, Monolith und Microservices. Kapitel \ref{lab:cloud} widmet sich dem Cloud Computing und beleuchtet die Grundlagen der Cloud-Technologie, insbesondere Cloud Native Plattformen und Kubernetes.
Kapitel \ref{lab:application_observability} führt in das Konzept der Application Observability ein, beginnend mit einer klaren Begriffsdefinition und einer Darstellung der grundlegenden Bausteine wie Protokolle und Log-Dateien, Metriken, Tracing und Telemetriedaten. Zudem werden die spezifischen Herausforderungen in Cloud Native Umgebungen beleuchtet.
Die methodische Vorgehensweise, beschrieben im Kapitel \ref{lab:methodische_vorgehensweise}, beginnt mit der Erläuterung der Wahl des Prototyping-Ansatzes. Anschließend erfolgt die Identifizierung der Kernanforderungen an eine Cloud Native Plattform. Die Planung und das Design werden ebenso detailliert vorgestellt.
Kapitel \ref{lab:prototyp} umfasst den Praxis Anteil der Thesis indem ein Prototyp des zuvor entwickelten Modells erstellt wird. Die gewonnenen Ergebnisse werden im Kapitel \ref{lab:ergebnisse} zusammengefasst und diskutiert. Kapitel \ref{lab:kritische_betrachtung} leitet die kritische Betrachtung ein. Hier werden die Limitationen der Untersuchung betrachtet in bezug auf die gewählte Methode und die erzielten Ergebnisse. Ein Ausblick auf mögliche zukünftige Forschungsansätze schließen das Kapitel ab. Kapitel \ref{lab:fazit} läutet das Ende der Thesis ein, umschließt das Fazit und fasst die wichtigsten Erkenntnisse der Arbeit zusammen. Schlussendlich abgeschlossen wird die Arbeit mit einem Ausblick.

