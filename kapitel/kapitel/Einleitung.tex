\newpage
\section{Einleitung}
\subsection{Ein Medienecho}
https://www.computerwoche.de/a/was-gegen-cloud-native-spricht,3613754
https://www.datacenter-insider.de/podcastmitsebastiankistervomkubernetes-kompetenz-zentrumbeiaudi-a-6276da9a37b65eed7af33b32199ecdbd/
https://www.it-business.de/cloud-native-ist-im-channel-angekommen-a-ea90aa27d988c6c0c5815aaacc56d360/
https://www.security-insider.de/cloud-native-security-vier-herausforderungen-und-drei-tipps-a-51f23e225b38678ae8ec8b38e10941ed/

Cloud Computing, 
\subsection{Problemstellung und Zielsetzung der Thesis}
\subsubsection{Problem}
Die rasante Entwicklung von Cloud-Technologien und die wachsende Nachfrage nach agilen und skalierbaren Anwendungen haben zu einer Revolution in der Art und Weise geführt, wie Unternehmen ihre Software entwickeln, bereitstellen und betreiben. Die traditionelle monolithische Architektur wird zunehmend von einer auf Microservices basierenden Architektur abgelöst, die es ermöglicht, Anwendungen in kleinere, eigenständige Komponenten zu zerlegen. Diese Komponenten, auch als containerisierte Microservices bezeichnet, können unabhängig voneinander entwickelt, bereitgestellt und skaliert werden. Die Cloud bietet eine ideale Umgebung für den Einsatz solcher Microservices, da sie elastische Ressourcen, automatische Skalierung und flexible Bereitstellungsmöglichkeiten bietet.

Der Einsatz von containerisierten Microservices in der Cloud bringt jedoch auch neue Herausforderungen mit sich. Eine entscheidende Frage ist die Auswahl geeigneter Technologien und Tools, um die Microservices effizient zu verwalten und zu orchestrieren. Container-Orchestrierungssysteme wie Kubernetes haben sich als Standard etabliert, bieten jedoch eine Vielzahl von Möglichkeiten und Konfigurationen, die sorgfältig abgewogen werden müssen. Die Gewährleistung von Skalierbarkeit ist ein weiterer wichtiger Aspekt. Eine erfolgreiche Cloud-Native-Plattform muss in der Lage sein, die Anwendungen dynamisch zu skalieren, um eine hohe Auslastung zu bewältigen und gleichzeitig Ressourcenverschwendung zu vermeiden.

Darüber hinaus dürfen die Sicherheitsaspekte nicht vernachlässigt werden. Eine Cloud-Native-Plattform muss Mechanismen zur Isolation und Absicherung der einzelnen Microservices bereitstellen, um die Vertraulichkeit, Integrität und Verfügbarkeit der Daten zu gewährleisten. Ebenso spielt die effiziente Ressourcennutzung eine zentrale Rolle, da eine effektive Verwaltung und Auslastung der Ressourcen in der Cloud entscheidend für die Wirtschaftlichkeit und Leistungsfähigkeit der Plattform ist.

Vor diesem Hintergrund wird deutlich, dass der architektonische Entwurf und die Modellierung einer geeigneten Cloud-Native-Plattform von entscheidender Bedeutung sind. Eine fundierte und gut durchdachte Architektur legt den Grundstein für den erfolgreichen Einsatz von containerisierten Microservices und Anwendungen in der Cloud. Durch die Schaffung einer skalierbaren, sicheren und effizienten Plattform können Unternehmen die Vorteile der Cloud voll ausschöpfen und ihre Anwendungen mit Agilität und Skalierbarkeit entwickeln und betreiben.
\\
Das Ziel dieser Arbeit ist es, einen architektonischen Entwurf und eine Modellierung einer Cloud-Native-Plattform für den Einsatz von containerisierten Microservices und Anwendungen zu entwickeln. Dabei sollen die spezifizierten Anforderungen berücksichtigt werden, die mit der Skalierbarkeit und effizienten Ressourcennutzung verbunden sind. Die Arbeit wird verschiedene Aspekte der Architektur und Modellierung betrachten, wie die Auswahl geeigneter Technologien, sowie die Integration von Monitoring und Management-Funktionalitäten. Durch den architektonischen Entwurf und die Modellierung einer Cloud-Native-Plattform sollen Unternehmen in der Lage sein, die Vorteile von containerisierten Microservices und Anwendungen in der Cloud optimal zu nutzen und ihre IT-Infrastruktur effektiv zu modernisieren.
\subsection{Abgrenzung}
keine Sicherheit und authentifizierung
\subsection{Methodik und Vorgehensweise}